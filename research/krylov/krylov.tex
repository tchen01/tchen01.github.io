\hypertarget{motivation}{%
\subsection{Motivation}\label{motivation}}

Solving a linear system of equations \(Ax=b\) is one of the most
important tasks in modern science. A huge number of techniques and
algorithms for dealing with more complex equations end up using linear
approximations. As a result, applications such as weather forecasting,
medical imaging, and training neural nets all require repeatedly solving
linear systems to achieve the real world impact that we often take for
granted. When \(A\) is symmetric and positive definite (if you don't
remember what that means, don't worry, I have a refresher below), the
Conjugate Gradient algorithm is a very popular choice for methods of
solving \(Ax=b\).

This popularity of CG is due to a couple factors. First, like most
Krylov subspace methods, CG is \emph{matrix free}. This means that you
don't ever need to explicitly represent \(A\) as a matrix, you only need
to be able to compute the product \(v\mapsto Av\) for a given input
vector \(v\). For very large problems this means a big reduction in
storage, and if \(A\) has some structure (eg, \(A\) comes from a DFT,
difference/integral operator, is very sparse, etc.), it allows the
algorithm to take advantage of fast matrix vector products. Second, CG
only requires \(\mathcal{O}(n)\) additional storage to run (as compared
to \(\mathcal{O}(n^2)\) that many other algorithms require). This can be
very useful when the size of the system is very large as it reduces the
communication costs of moving data in and out of memory/caches.

\hypertarget{measuring-the-accuracy-of-solutions}{%
\subsection{Measuring the accuracy of
solutions}\label{measuring-the-accuracy-of-solutions}}

Perhaps the first question that should be asked about any numerical
method is , \emph{does it solve the intended problem?} In the case of
solving linear systems, this means asking \emph{does the output
approximate the true solution?} If not, then there isn't much point
using the method.

Let's quickly introduce the idea of the \emph{error} and the
\emph{residual}. These quantities are both useful (in different ways)
for measuring how close the approximate solution \(\tilde{x}\) is to the
true solution \(x^* = A^{-1}b\).

The \emph{error} is simply the difference between \(x\) and
\(\tilde{x}\). Taking the norm of this quantity gives us a scalar value
which measures the distance between \(x\) and \(\tilde{x}\). In some
sense, this is perhaps the most natural way of measuring how close our
approximate solution is to the true solution. In fact, when we say the
sequence \(x_0,x_1,x_2,\ldots\) converges to \(x_*\), we mean that the
scalar sequence,\(\|x^*-x_0\|,\|x^*-x_1\|,\|x^*-x_2\|,\ldots\) converges
to zero. Thus, solving \(Ax=b\) could be written as minimizing
\(\|x - x^*\| = \|x-A^{-1}b\|\).

Of course, since we are trying to compute \(x^*\), it doesn't make sense
for an algorithm to explicitly depend on \(x^*\). The \emph{residual} of
\(\tilde{x}\) is defined as \(b-A\tilde{x}\). Again, \(\|b-Ax^*\| = 0\),
and since \(x^*\) is the only point where this is true, minimizing
\(\|b-Ax\|\) gives the true solution. The advantage is that we can
easily compute the residual \(b-A\tilde{x}\) once we have our numerical
solution \(\tilde{x}\), while there is not necessarily a good way to
compute the error \(x^*-x\).

\hypertarget{krylov-subspaces}{%
\subsection{Krylov subspaces}\label{krylov-subspaces}}

From the previous section, we know that minimizing \(\|b-Ax\|\) will
give the solution \(x^*\). Unfortunately, this problem is ``just as
hard'' as solving \(Ax=b\).

We would like to relax this problem in some way to make it ``easier''.
One way to do this is to restrict the values that \(x\) can be. For
instance, we can enforce that \(x\) comes from a smaller set of values
which should make the problem of minimizing \(\|b-Ax\|\) simpler (since
there are less possibilities for \(x\)). For instance, if we say that
\(x = cy\) for some fixed vector \(y\), then this is a scalar
minimization problem. Of course, by restricting what values we choose
for \(x\) it is quite likely that we will not longer be able to exactly
solve \(Ax=b\).

It then makes sense to try to balance the difficulty of the problems we
have to solve at each step with the accuracy of the solutions they give.
One way to do this is to start with an easy problem and get a very
approximate solution, and then gradually increase the difficulty of the
problem while refining the solution. If we do it in the right way, it's
plausible that ``increasing the difficulty'' of the problem we are
solving won't lead to extra work at each step, since we might be able to
take advantage of having an approximate solution from a previous step.

Suppose we have a sequence of subspaces
\(V_0\subset V_1\subset V_2\subset \cdots V_m\). Then we can construct a
sequence of iterates, \(x_0\in V_0, x_1\in V_1,\ldots\). If, at each
step, we ensure that \(x_k\) minimizes \(\|b-Ax\|\) over \(V_k\), then
the norm of the residuals will decrease (because
\(V_k \subset V_{k+1}\)).

Ideally this sequences of subspaces would:

\begin{enumerate}
\def\labelenumi{\arabic{enumi}.}
\tightlist
\item
  be easy to construct
\item
  be easy to optimize over (given the previous work done)
\item
  eventually contain the true solution
\end{enumerate}

We now formally introduce Krylov subspaces, and show that they can
satisfy these properties.

The \(k\)-th Krylov subspace generated by a square matrix \(A\) and a
vector \(v\) is defined to be, \[
\mathcal{K}_k(A,v) = \operatorname{span}\{v,Av,\ldots,A^{k-1}v \}
\]

First, these subspaces are relatively easy to construct because we can
get a basis by repeatedly applying \(A\) to \(v\). In fact, we can
fairly easily construct an orthonormal basis for these spaces with the
\href{./arnoldi_lanczos.html}{Arnoldi/Lanczos} algorithms.

Therefore, if we can find a quantity which can be optimized over each
direction of an orthonormal basis independently, then optimizing over
these expanding subspaces will be easy because we only need to optimize
in a single new direction at each step.

We now show that \(\mathcal{K}_k(A,b)\) will eventually contain our
solution by the time \(k=n\). While this result comes about naturally
from our derivation of CG, I think it is useful to relate polynomials
with Krylov subspace methods early on, as the two are intimately
related.

Suppose \(A\) has
\href{https://en.wikipedia.org/wiki/Characteristic_polynomial\#Characteristic_equation}{characteristic
polynomial}, \[
p_A(t) = \det(tI-A) = c_0 + c_1t + \cdots + c_{n-1}t^{n-1} + t^n
\] It turns out that \(c_0 = (-1)^n\det(A)\) so that \(c_0\) is nonzero
if \(A\) is invertible.

The
\href{https://en.wikipedia.org/wiki/Cayley\%E2\%80\%93Hamilton_theorem}{Cayley-Hamilton
Theorem} states that a matrix satisfies its own characteristic
polynomial. This means, \[
0 = p_A(A) = c_0 I + c_1 A + \cdots c_{n+1} A^{n-1} + A^n
\]

Moving the identity term to the left and dividing by \(-c_0\) (which
won't be zero since \(A\) is invertible) we can write, \[
A^{-1} = -(c_1/c_0) I - (c_2/c_0) A - \cdots - (1/c_0) A^{n-1}
\]

This says that \(A^{-1}\) can be written as a polynomial in \(A\)! (I
think the coolest facts from linear algebra.) In particular,\\
\[
x^* = A^{-1}b = -(c_1/c_0) b - (c_2/c_0) Ab - \cdots - (1/c_0) A^{n-1}b
\]

That is, the solution \(x^*\) to the system \(Ax = b\) is a linear
combination of \(b, Ab, A^2b, \ldots, A^{n-1}b\) (i.e.
\(x^*\in\mathcal{K}_n(A,b)\)). This observation is the motivation behind
Krylov subspace methods. I might be useful to think of Krylov subspace
methods as building low degree polynomial approximations to \(A^{-1}b\)
using powers of \(A\) times \(b\) (in fact Krylov subspace methods can
be used to approximate \(f(A)b\) where \(f\) is any
\href{./current_research.html}{function}).

\hypertarget{the-arnoldi-and-lanczos-algorithms}{%
\section{The Arnoldi and Lanczos
algorithms}\label{the-arnoldi-and-lanczos-algorithms}}

The Arnoldi and Lanczos algorithms for computing an orthonormal basis
for Krylov subspaces are, in one way or another, at the core of all
Krylov subspace methods. Essentially, these algorithms are the
Gram-Schmidt procedure applied to the vectors \(v,Av,A^2v,A^3v,\ldots\)
in a clever way.

\hypertarget{the-arnoldi-algorithm}{%
\subsection{The Arnoldi algorithm}\label{the-arnoldi-algorithm}}

Recall that given a set of vectors \(v_1,v_2,\ldots, v_k\) the
Gram-Schmidt procedure computes an orthonormal basis
\(q_1,q_2,\ldots,q_k\) so that for all \(j\leq k\), \[
\operatorname{span}\{v_1,\ldots,v_j\} = \operatorname{span}\{q_1,\ldots,q_j\}
\]

The trick behind the Arnoldi algorithm is the fact that you do not need
to construct the whole set \(v,Av,A^2v,\ldots\) ahead of time (in
practice, if you tried to do this, it wouldn't really work because
eventually \(A^jv\) and \(A^{j+1}v\) will be nearly linearly dependent
since this is essentially the
\href{https://en.wikipedia.org/wiki/Power_iteration}{power method}).
Instead, you can compute \(Aq_{k}\) in place of \(A^{k+1}v\) once you
have found an orthonormal basis \(q_1,q_2,\ldots,q_k\) spanning
\(v,Av,\ldots, A^{k-1} v\).

If we assume that
\(\operatorname{span}\{v,Av,\ldots A^{k-1} v\}= \operatorname{span}\{q_1,\ldots, q_k\}\)
then \(q_{k+1}\) can be written as a linear combination of
\(v,Av,\ldots, A^k v\). Therefore, \(Aq_k\) will be a linear combination
of \(Av,A^2v,\ldots,A^k v\). In particular, this means that
\(\operatorname{span}\{q_1,\ldots,q_k,Aq_k\} = \operatorname{span}\{v,Av,\ldots,A^k v\}\).
Therefore, we will get exactly the same set of vectors by applying
Gram-Schmidt to \(\{v,Av,\ldots,A^kv\}\) as if we compute \(Aq_k\) once
we have computing \(q_k\).

Since we obtain \(q_{k+1}\) by orthogonalizing \(Aq_k\) against
\(\{q_1,q_2,\ldots,q_k\}\) then \(q_{k+1}\) is in the span of these
vectors, there exist some \(c_i\) so that, \[
q_{k+1} = c_1 q_1 + c_2 q_2 + \cdots + c_k q_k + c_{k+1}Aq_k
\]

We can rearrange this (using new scalars \(d_i\)) to, \[
Aq_k = d_1q_1 + d_2q_2 + \cdots + d_{k+1} q_{k+1}
\]

This can be written in matrix form as, \[
AQ = QH
\] where \(H\) is ``upper Hessenburg'' (like upper triangular but the
first subdiagonal also has nonzero entries). While I'm not going to
derive them here, since the entries of \(H\) come directly from the
Arnoldi algorithm (just like how the entries of \(R\) in a QR
factorization can be obtained from Gram Schmidt) their explicit
expressions can be easily written down.

Since \(Q\) is orthogonal then, \(Q^*AQ = H\), so \(H\) and \(A\) are
similar. This means that finding the eigenvalues and vectors of \(H\)
will give us the eigenvalues and vectors of \(A\). However, since \(H\)
is upper Hessenburg, then solving the eigenproblem is easier than for a
general matrix.

\hypertarget{the-lancozs-algorithm}{%
\subsection{The Lancozs algorithm}\label{the-lancozs-algorithm}}

When \(A\) is Hermetian, then \(Q^*AQ = H\) is also Hermetian. Since
\(H\) is upper Hessenburg and Hermitian, it must be tridiagonal! This
means that the \(q_j\) satisfy a three term recurrence, \[
Aq_j = \beta_{j-1} q_{j-1} + \alpha_j q_j + \beta_j q_{j+1}
\] where \(\alpha_1,\ldots,\alpha_n\) are the diagonal entries of \(T\)
and \(\beta_1,\ldots,\beta_{n-1}\) are the off diagonal entries of
\(T\). The Lanczos algorithm is an efficient way of computing this
decomposition.

I will present a brief derivation for the method motivated by the three
term recurrence above. Since we know that the \(q_j\) satisfy the three
term recurrence, we would like the method to store as few of the \(q_j\)
as possible (i.e.~take advantage of the three term recurrence as opposed
to the Arnoldi algorithm).

Suppose that we have \(q_j\), \(q_{j-1}\), and the coefficient
\(\beta_{j-1}\), and want expand the Krylov subspace to find \(q_{j+1}\)
in a way that takes advantage of the three term recurrence. To do this
we can expand the subspace by computing \(Aq_j\) and then
orthogonalizing \(Aq_j\) against \(q_j\) and \(q_{j-1}\). By the three
term recurrence, \(Aq_j\) will be orthogonal to \(q_i\) for all
\(i\leq j-2\) so we do not need to explicitly orthogonalize against
those vectors.

We orthogonalize, \begin{align*}
\tilde{q}_{j+1} = Aq_j - \alpha_j q_j - \langle Aq_j, q_{j-1} \rangle q_{j-1}
, && 
\alpha_{j} = \langle A q_j, q_j \rangle
\end{align*} and finally normalize, \begin{align*}
q_{j+1} = \tilde{q}_{j+1} / \beta_j
,&&
\beta_j = \|\tilde{q}_{j+1}\|
\end{align*}

Note that this is not the most ``numerically stable'' form of the
algorithm, and care must be taken when implementing the Lanczos method
in practice. We can improve stability slightly by using
\(Aq_j - \beta_{j-1} q_{j-1}\) instead of \(Aq_j\) when finding a vector
in the next Krylov subspace. This allows us to ensure that we have
orthogonalized \(q_{j+1}\) against \(q_j\) and \(q_{j-1}\) rather than
just \(q_j\). It also ensures that the tridiagonal matrix produces is
symmetric in finite precision (since \(\langle Aq_j,q_{j-1}\rangle\) may
not be equal to \(\beta_j\) in finite precision).

We can \href{./lanczos.py}{implement} Lanczos iteration in numpy. Here
we assume that we only want to output the diagonals of the tridiagonal
matrix \(T\), and don't need any of the vectors (this would be useful if
we wanted to compute the eigenvalues of \(A\), but not the
eigenvectors).

\begin{verbatim}
def lanczos(A,q0,max_iter):
    alpha = np.zeros(max_iter)
    beta = np.zeros(max_iter)
    q_ = np.zeros(len(q0))
    q = q0/np.sqrt(q0@q0)

    for k in range(max_iter):
        qq = A@q-(beta[k-1]*q_ if k>0 else 0)
        alpha[k] = qq@q
        qq -= alpha[k]*q
        beta[k] = np.sqrt(qq@qq)
        q_ = np.copy(q)
        q = qq/beta[k]

return alpha,beta
\end{verbatim}

\hypertarget{a-derivation-of-the-conjugate-gradient-algorithm}{%
\section{A derivation of the Conjugate Gradient
algorithm}\label{a-derivation-of-the-conjugate-gradient-algorithm}}

There are many ways to view/derive the Conjugate Gradient algorithm.
I'll derive the algorithm by directly minimizing by minimizing the
\(A\)-norm of the error over successive Krylov subspaces,
\(\mathcal{K}_k(A,b)\), which I think is the most natural way to view
the algorithm. My hope is that the derivation here provides an intuitive
introduction to CG. Of course, what I think is a good way to present the
topic won't match up exactly with every reader's own preference, so I
highly recommend looking through some other resources as well. To me,
this is of those topics where you have to go through the explanations a
few times before you start to understand what is going on.

\hypertarget{linear-algebra-review}{%
\subsection{Linear algebra review}\label{linear-algebra-review}}

Before we get into the details, let's define some notation and review a
few key concepts from linear algebra which we will rely on when deriving
the CG algorithm.

\begin{itemize}
\tightlist
\item
  Any inner product \(\langle \cdot,\cdot \rangle\) induces a norm
  \(\|\cdot\|\) defined by \(\|x\|^2 = \langle x,x\rangle\).
\item
  For the rest of this piece we will denote the standard (Euclidian)
  inner product by \(\langle \cdot,\cdot\rangle\) and the (Euclidian)
  norm by \(\|\cdot\|\) or \(\|\cdot\|_2\).
\item
  A matrix \(A\) is positive definite if \(\langle x, Ax\rangle > 0\)
  for all \(x\).
\item
  A symmetric positive definite matrix \(A\) naturally induces the inner
  product \(\langle \cdot,\cdot \rangle_A\) defined by
  \(\langle x,y\rangle_A = \langle x,Ay\rangle = \langle Ax,y \rangle\).
  The associated norm, called the \(A\)-norm will be denoted by
  \(\|langle \cdot \|_A\) and is defined by, \[
  \|x\|_A^2 = \langle x,x \rangle_A = \langle x,Ax \rangle = \| A^{1/2}x \|
  \]
\item
  The point in a subspace \(V\) nearest to a point \(x\) is the
  projection of \(x\) onto \(V\) (where projection is done with the
  inner product and distance is measured with the induced norm). Given
  an orthonormal basis for \(V\), this amounts to summing the projection
  of \(x\) onto each of the basis vectors.
\item
  The \(k\)-th Krylov subspace generated by \(A\) and \(b\) is, \[
  \mathcal{K}_k(A,b) = \operatorname{span}\{b,Ab,\ldots,A^{k-1}b\}
  \]
\end{itemize}

\hypertarget{minimizing-the-error}{%
\subsection{Minimizing the error}\label{minimizing-the-error}}

Now that we have that out of the way, let's begin our derivation. As
stated above, we will minimize the \(A\)-norm of the error over
successive Krylov subspaces generated by \(A\) and \(b\). That is to say
\(x_k\) will be the point so that, \begin{align*}
\|e_k\|_A
:=\| x_k - x^* \|_A 
= \min_{x\in\mathcal{K}_k(A,b)} \| x - x^* \|_A
,&&
x^* = A^{-1}b
\end{align*}

Since we are minimizing with respect to the \(A\)-norm, it will be
useful to have an \(A\)-orthonormal basis for \(\mathcal{K}_k(A,b)\).
That is, a basis which is orthonormal in the \(A\)-inner product. For
now, let's just say we have such a basis,
\(\{p_0,p_1,\ldots,p_{k-1}\}\), ahead of time. Since
\(x_k\in\mathcal{K}_k(A,b)\) we can write \(x_k\) in terms of this
basis, \[
x_k = a_0 p_0 + a_1 p_1 + \cdots + a_{k-1} p_{k-1}
\]

Note that we have \(x_0 = 0\) and \(e_k = x^* - x_k\). Then, \[
e_k = e_0 - a_0p_0 - a_1 p_1 - \cdots - a_{k-1} p_{k-1}
\]

By definition, the coefficients for \(x_k\) were chosen to minimize the
\(A\)-norm of the error, \(\|e_k\|_A\), over \(\mathcal{K}_k(A,b)\).
Therefore, \(e_k\) has zero component in each of the directions
\(\{ p_0,p_1,\ldots,p_{k-1} \}\). In particular, that means that
\(a_jp_j\) cancels exactly with \(e_0\) in the direction of \(p_j\).

We now make an important observation. Namely, that the coefficients do
not depend on \(k\). Therefore, since the coefficients
\(a_0',a_1',\ldots,a_{k-2}'\) of \(x_{k-1}\) were chosen in exactly the
same way as the coefficients for \(x_k\), then
\(a_0=a_0', a_1=a_1', \ldots, a_{k-2}=a_{k-2}'\).

We can then write, \[
x_k = x_{k-1} + a_{k-1} p_{k-1}
\] and \[
e_k = e_{k-1} - a_{k-1} p_{k-1}
\]

Now that we have explicitly written \(x_k\) in terms of an update to
\(x_{k-1}\) this is starting to look like an iterative method!

Let's compute an explicit representation of the coefficient \(a_{k-1}\).
As previously noted, we have chosen \(x_k\) to minimize \(\|e_k\|_A\)
over \(\mathcal{K}_k(A,b)\). Therefore, the component of \(e_k\) in each
of the directions \(p_0,p_1,\ldots,p_{k-1}\) must be zero. That is,
\(\langle e_k , p_j \rangle = 0\) for all \(i=0,1,\ldots, k-1\). \[
0 = \langle e_k , p_{k-1} \rangle_A
= \langle e_{k-1}, p_{k-1} \rangle - a_{k-1} \langle p_{k-1} , p_{k-1} \rangle_A
\]

Thus \[
a_{k-1} 
= \frac{\langle e_{k-1}, p_{k-1} \rangle_A}{\langle p_{k-1},p_{k-1} \rangle_A} 
\]

This expression might look like a bit of a roadbock, since if we knew
the initial error \(e_0 = x^* - 0\) then we would know the solution to
the original system! However, we have been working with the \(A\)-inner
product so we can write, \[
Ae_{k-1} = A(x^* - x_{k-1}) = b - Ax_{k-1} = r_{k-1}
\] Therefore, we can compute \(a_{k-1}\) as, \[
a_{k-1}
= \frac{\langle r_{k-1}, p_{k-1} \rangle}{\langle p_{k-1},A p_{k-1} \rangle} 
\]

\hypertarget{finding-the-search-directions}{%
\subsection{Finding the Search
Directions}\label{finding-the-search-directions}}

At this point we are almost done. The last thing to do is understand how
to update \(p_k\). The first thing we might try would be to do something
like Gram-Schmidt on \(\{b,Ab,A^2b,\ldots \}\) to get the \(p_k\),
i.e.~Arnoldi iteration in the inner product induced by \(A\). This will
work fine if you take some care with the exact implementation. However,
since \(A\) is symmetric we might hope to be able to use some short
recurrence, which turns out to be the case.

Since \(r_k = b-Ax_k\) and \(x_k\in\mathcal{K}_k(A,b)\), then
\(r_k \in \mathcal{K}_{k+1}(A,b)\). Thus, we can obtain \(p_k\) by
\(A\)-orthogonalizing \(r_k\) against \(\{p_0,p_1,\ldots,p_{k-1}\}\).

Recall that \(e_k\) is \(A\)-orthogonal to \(\mathcal{K}_k(A,b)\). That
is, for \(j\leq k-1\), \[
\langle e_k, A^j b \rangle_A = 0
\]

Therefore, noting that \(Ae_k = r_k\), for \(j\leq k-2\), \[
\langle r_k, A^j b \rangle_A = 0
\]

That is, \(r_k\) is \(A\)-orthogonal to \(\mathcal{K}_{k-1}(A,b)\). In
particular, this means that, for \(j\leq k-2\), \[
\langle r_k, p_j \rangle_A = 0
\]

That means that to obtain \(p_k\) we really only need to
\(A\)-orthogonalize \(r_k\) against \(p_{k-1}\)! That is, \begin{align*}
p_k = r_k + b_k p_{k-1}
,&&
b_k = - \frac{\langle r_k, p_{k-1} \rangle_A}{\langle p_{k-1}, p_{k-1} \rangle_A}
\end{align*}

The immediate consequence is that we do not need to save the entire
basis \(\{p_0,p_1,\ldots,p_{k-1}\}\), but instead can just keep
\(x_k\),\(r_k\), and \(p_{k-1}\). \textbf{expand on this}!!

\hypertarget{putting-it-all-together}{%
\subsection{Putting it all together}\label{putting-it-all-together}}

We are now essentially done! In practice, people generally use the
following equivalent formulas for \(a_{k-1}\) and \(b_k\),
\begin{align*}
a_{k-1} = \frac{\langle r_{k-1},r_{k-1}\rangle}{\langle p_{k-1},Ap_{k-1}\rangle}
,&&
b_k = \frac{\langle r_k,r_k\rangle}{\langle r_{k-1},r_{k-1}\rangle}
\end{align*}

The first people to discover this algorithm Magnus Hestenes and Eduard
Stiefel who independently developed it around 1952. As such, the
standard implementation is attributed to them.

\textbf{Algorithm.} (Hestenes and Stiefel Conjugate Gradient)
\begin{align*}
&\textbf{procedure}\text{ HSCG}( A,b,x_0 ) 
\\[-.4em]&~~~~\textbf{set } r_0 = b-Ax_0, \nu_0 = \langle r_0,r_0 \rangle, p_0 = r_0, s_0 = Ar_0, 
\\[-.4em]&~~~~\phantom{\textbf{set }}a_0 = \nu_0 / \langle p_0,s_0 \rangle
\\[-.4em]&~~~~\textbf{for } k=1,2,\ldots \textbf{:} 
\\[-.4em]&~~~~~~~~\textbf{set } x_k = x_{k-1} + a_{k-1} p_{k-1} 
\\[-.4em]&~~~~~~~~\phantom{\textbf{set }} r_k = r_{k-1} - a_{k-1} p_{k-1} 
\\[-.4em]&~~~~~~~~\textbf{set } \nu_{k} = \langle r_k,r_k \rangle, \textbf{ and } b_k = \nu_k / \nu_{k-1}
\\[-.4em]&~~~~~~~~\textbf{set }p_k = r_k + b_k p_{k-1}
\\[-.4em]&~~~~~~~~\textbf{set }s_k = A p_k
\\[-.4em]&~~~~~~~~\textbf{set }\mu_k = \langle p_k,s_k \rangle, \textbf{ and } a_k = \nu_k / \mu_k
\\[-.4em]&~~~~~\textbf{end for}
\\[-.4em]&\textbf{end procedure}
\end{align*}

This can be easily \href{./cg.py}{implemented} in numpy. Note that we
use \(f\) for the right hand side vector to avoid conflict with the
coefficient \(b\).

\begin{verbatim}
def cg(A,f,max_iter):
    x = np.zeros(len(f)); r = np.copy(f); p = np.copy(r); s=A@p
    nu = r @ r; a = nu/(p@s); b = 0
    for k in range(1,max_iter):
        x += a*p
        r -= a*s

        nu_ = nu
        nu = r@r
        b = nu/nu_

        p = r + b*p
        s = A@p

        a = nu/(p@s)

    return x
\end{verbatim}

\hypertarget{conjugate-gradient-is-lanczos-in-disguise}{%
\section{Conjugate Gradient is Lanczos in
Disguise}\label{conjugate-gradient-is-lanczos-in-disguise}}

It's perhaps not so surprising that the Conjugate Gradient and Lanczos
algorithms are closely related. After all, they are both Krylov subspace
methods for symmetric matrices.

More precisely, the Lanczos algorithm will produce an orthonormal basis
for \(\mathcal{K}_k(A,b)\), \(k=0,1,\ldots\) if we initialize with
initial vector \(r_0 = b\). We also know that the Conjugate Gradient
residuals form an orthogonal basis for for these spaces, which means
that the Lanczos vectors must be scaled versions of the Conjugate
Gradient residuals.

This relationship provides a way of transferring research about the
Lanczos algorithm to be to CG, and visa versa. In fact, the analysis of
\href{./finite_precision_cg.html}{finite precision CG} done by Greenbaum
requires viewing CG in terms of the Lanczos recurrence.

In case you're just looking for the punchline, the Lanczos coefficients
and vectors can be obtained from the Conjugate Gradient algorithm by the
following relationship, \begin{align*}
q_{j+1} \equiv (-1)^j\dfrac{r_j}{\|r_j\|}
,&&
\beta_j \equiv \frac{\|r_j\|}{\|r_{j-1}\|}\frac{1}{a_{j-1}}
,&&
\alpha_j \equiv \left(\frac{1}{a_{j-1}} + \frac{b_{j}}{a_{j-2}}\right)
\end{align*}

Note that the indices are offset, because the Lanczos algorithm is
started with initial vector \(q_1\).

\hypertarget{derivation}{%
\subsection{Derivation}\label{derivation}}

The derivation of the above result is so much difficult as it is
tedious. Before you read my derivation, I would highly recommend trying
to derive it on your own, since it's a good chance to improve your
familiarity with both algorithms. While I hope that my derivation is not
too hard to follow, there are definitely other ways to arrive at the
same result, and often to really start to understand something you have
to work it out on your own.

Recall that the three term Lanczos recurrence is, \begin{align*}
Aq_j = \alpha_j q_j + \beta_{j-1}q_{j-1} + \beta_j q_{j+1}
\end{align*}

In each iteration of CG we update, \begin{align*}
r_j = r_{j-1} - a_{j-1} Ap_{j-1}
,&&
p_j = r_j + b_j p_{j-1}
\end{align*}

Thus, substituting the expression for \(p_{j-1}\) we find,
\begin{align*}
r_j &= r_{j-1} - a_{j-1} A(r_{j-1} + b_{j-1} p_{j-2})
\\&= r_{j-1} - a_{j-1} Ar_{j-1} - a_{j-1}b_{j-1} A p_{j-2}
\end{align*}

Now, rearranging our equation for \(r_{j-1}\) we have that
\(Ap_{j-2} = (r_{j-2} - r_{j-1}) / a_{j-2}\). Therefore, \begin{align*}
    r_j &= r_{j-1} - a_{j-1} Ar_{j-1} - \frac{a_{j-1}b_{j-1}}{a_{j-2}}(r_{j-2} - r_{j-1})
\end{align*}

At this point we've found a three term recurrence for \(r_j\), which is
a hopeful sign that we are on the right track. We know that the Lanczos
vectors are orthonormal and that the recurrence is symmetric, so we'll
keep massaging our CG relation to try to get it into that form.

First, let's rearrange terms and regroup them so that the indices and
matrix multiply match up with the Lanczos recurrence. This gives,
\begin{align*}
    Ar_{j-1} = \left(\frac{1}{a_{j-1}}+\frac{b_{j-1}}{a_{j-2}}\right) r_{j-1} - \frac{b_{j-1}}{a_{j-2}} r_{j-2} - \frac{1}{a_{j-1}} r_{j}
\end{align*}

Now, we normalize our residuals as \(z_j = r_{j-1}/\|r_{j-1}\|\) so that
\(r_{j-1} = \|r_{j-1}\| z_j\). Plugging these in gives, \begin{align*}
    \|r_{j-1}\|Az_{j} &= \|r_{j-1}\|\left(\frac{1}{a_{j-1}}+\frac{b_{j-1}}{a_{j-2}}\right) z_{j} 
    -\|r_{j-2}\|\frac{b_{j-1}}{a_{j-2}} z_{j} - \|r_j\| \frac{1}{a_{j-1}} z_{j+1}
\end{align*}

Thus, dividing through by \(\|r_{j-1}\|\) we have, \begin{align*}
    Az_{j} &= \left(\frac{1}{a_{j-1}}+\frac{b_{j-1}}{a_{j-2}}\right) z_{j} 
    - \frac{\|r_{j-2}\|}{\|r_{j-1}\|}\frac{b_{j-1}}{a_{j-2}} z_{j-1} - \frac{\|r_j\|}{\|r_{j-1}\|} \frac{1}{a_{j-1}} z_{j+1}
\end{align*}

This looks close, but the coefficients for the last two terms have the
same formula in the Lanczos recurrence. However, recall that
\(b_{j} = \langle r_j,r_j \rangle / \langle r_{j-1},r_{j-1} \rangle = \|r_j\|^2 / \|r_{j-1}\|^2\).
Thus, \begin{align*}
    Az_{j} &= \left(\frac{1}{a_{j-1}}-\frac{b_{j-1}}{a_{j-2}}\right) z_{j} 
    - \frac{\|r_{j-1}\|}{\|r_{j-2}\|}\frac{1}{a_{j-2}} z_{j-1} - \frac{\|r_j\|}{\|r_{j-1}\|} \frac{1}{a_{j-1}} z_{j+1}
\end{align*}

We're almost there! While we have the correct for the recurrence, the
coefficients from the Lanczos method are always positive. This means
that our \(z_{j}\) have the wrong signs. Fixing this gives the
relationship, \begin{align*}
q_{j+1} \equiv (-1)^j\dfrac{r_j}{\|r_j\|}
,&&
\beta_j \equiv \frac{\|r_j\|}{\|r_{j-1}\|}\frac{1}{a_{j-1}}
,&&
\alpha_j \equiv \left(\frac{1}{a_{j-1}} + \frac{b_{j}}{a_{j-2}}\right)
\end{align*}

\hypertarget{error-bounds-for-the-conjugate-gradient-algorithm}{%
\section{Error Bounds for the Conjugate Gradient
Algorithm}\label{error-bounds-for-the-conjugate-gradient-algorithm}}

This page is a work in progress.

In our \href{./cg_derivation.html}{derivation} of the Conjugate Gradient
method, we minimized the \(A\)-norm of the error over sucessive Krylov
subspaces. Ideally we would like to know how quickly this method
converge. That is, how many iterations are needed to reach a specified
level of accuracy.

\hypertarget{linear-algebra-review-1}{%
\subsection{Linear algebra review}\label{linear-algebra-review-1}}

\begin{itemize}
\tightlist
\item
  The 2-norm of a symmetric positive definite matrix is the largest
  eigenvalue of the matrix
\item
  The 2-norm is submultiplicative. That is, \(\|A\|\|B\|\leq \|AB\|\)
\item
  A matrix \(U\) is called unitary if \(U^*U = UU^* = I\).
\end{itemize}

\hypertarget{polynomial-error-bounds}{%
\subsection{Polynomial error bounds}\label{polynomial-error-bounds}}

Previously we have show that, \[
e_k \in e_0 +  \operatorname{span}\{p_0,p_1,\ldots,p_{k-1}\} = e_0 + \mathcal{K}_k(A,b)
\]

Observing that \(r_0 = Ae_0\) we find that, \[
e_k \in e_0 +  \operatorname{span}\{Ae_0,A^2e_0,\ldots,A^{k}e_0\}
\]

Thus, we can write, \begin{align*}
\| e_k \|_A =  \min_{p\in\mathcal{P}_k}\|p(A)e_0\|_A
,&&
\mathcal{P}_k = \{p : p(0) = 1, \operatorname{deg} p \leq k\}    
\end{align*}

Since \(A^{1/2} p(A) = p(A)A^{1/2}\) we can write, \[
\| p(A)e_0 \|_A
= \|A^{1/2} p(A)e_0 \|
= \|p(A) A^{1/2}e_0 \|
\]

Now, using the submultiplicative property of the 2-norm, \[
\|p(A) A^{1/2}e_0 \|
\leq \|p(A)\| \|A^{1/2} e_0 \|
= \|p(A)\| \|e_0\|_A
\]

Since \(A\) is positive definite, it is diagonalizable as
\(U\Lambda U^*\) where \(U\) is unitary and \(\Lambda\) is the diagonal
matrix of eigenvalues of \(A\). Thus, \[
A^k = (U\Lambda U^*)^k = U\Lambda^kU^*
\]

We can then write \(p(A) = Up(\Lambda)U^*\) where \(p(\Lambda)\) has
diagonal entries \(p(\lambda_i)\). Therefore, using the \emph{unitary
invariance} property of the 2-norm, \[
\|p(A)\| = \|Up(\Lambda)U^*\| = \|p(\Lambda)\|
\]

Now, since the 2-norm of a symmetric matrix is the magnitude of the
largest eigenvalue, \[
\| p(\Lambda) \| = \max_i |p(\lambda_i)|
\]

Finally, putting everything together we have, \[
\frac{\|e_k\|_A}{\|e_0\|_A} \leq \min_{p\in\mathcal{P}_k} \max_i |p(\lambda_i)|
\]

Since the inequality we obtained from the submultiplicativity of the
2-norm is tight, this bound is also tight in the sense that for a fixed
\(k\) there exists an initial error \(e_0\) so that equality holds.

Computing the optimal \(p\) is not trivial, but an algorithm called the
\href{./remez.html}{Remez algorithm} can be used to compute it.

Let \(L\subset \mathbb{R}\) be some closed set. The \emph{minimax
polynomial of degree \(k\)} on \(L\) is the polynomial satisfying,
\begin{align*}
\min_{p\in\mathcal{P}_k} \max_{x\in L} | p(x) |
,&&
\mathcal{P}_k = \{p : p(0)=1, \deg p \leq k\}    
\end{align*}

\hypertarget{chebyshev-bounds}{%
\subsubsection{Chebyshev bounds}\label{chebyshev-bounds}}

The minimax polynomial on the eigenvalues of \(A\) is a bit tricky to
work with. Although we can find it using the Remez algorithm, this is
somewhat tedious, and requires knowledge of the whole spectrum of \(A\).
We would like to come up with a bound which depends on less information
about \(A\). One way to obtain such a bound is to expand the set on
which we are looking for the minimax polynomial.

To this end, let
\(\mathcal{I} = [\lambda_{\text{min}},\lambda_{\text{max}}]\). Then,
since \(\lambda_i\in\mathcal{I}\), \[
\min_{p\in\mathcal{P}_k} \max_i |p(\lambda_i)| 
\leq \min_{p\in\mathcal{P}_k} \max_{x \in \mathcal{I}} |p(x)| 
\]

The right hand side requires that we know the largest and smallest
eigenvalues of \(A\), but doesn't require any of the ones between. This
means it can be useful in practice, since we can easily compute the top
and bottom eignevalues with the power method.

The polynomials satisfying the right hand side are called the
\emph{Chebyshev Polynomials} and can be easily written down with a
simple recurrence relation. If \(\mathcal{I} = [-1,1]\) then the
relation is, \begin{align*}
T_{k+1}(x) = 2xT_k(x) - T_{k-1}(x)
,&&
T_0=1
,&&
T_1=x    
\end{align*}

For \(\mathcal{I} \neq [-1,1]\), the above polynomials are simply
stretched and shifted to the interval in question.

Let \(\kappa = \lambda_{\text{max}} / \lambda_{\text{min}}\) (this is
called the condition number). Then, from properties of these
polynomials, \[
\frac{\|e_k\|_A}{\|e_0\|_A} \leq 2 \left( \frac{\sqrt{\kappa}-1}{\sqrt{\kappa}+1} \right)^k
\]

\hypertarget{the-conjugate-gradient-algorithm-in-finite-precision}{%
\section{The Conjugate Gradient Algorithm in Finite
Precision}\label{the-conjugate-gradient-algorithm-in-finite-precision}}

This page is a work in progress.

A key component of our derivations of the
\href{./arnoldi_lanczos.html}{Lanczos} and
\href{./cg_derivation.html}{Conjugate Gradient} methods was the
orthogonality of certain basis vectors. In finite precision, our
induction based arguments no longer hold, and so it's reasonable to
expect the algorithms will fail. That said, since you're reading about
these methods, they must somehow still be usable in practice. This turns
out to be the case, and both methods are widely used for eignevalue
problems and solving linear systems.

The first major progress in the analysis of the Lanczos algorithm was
done by Chris Paige, who characterized the behavior of the method in
finite precision. A similarly important analysis of Conjugate Gradient
was done by Anne Greenbaum in her 1989 paper,
\href{https://www.sciencedirect.com/science/article/pii/0024379589902851}{``Behavior
of slightly perturbed Lanczos and conjugate-gradient recurrences''}. A
big takeaway from Greenbaum's analysis is that the error bound from the
Chevyshev polynomials still holds in finite precision (to a close
approximation). My goal here is to present the highlights of that paper.

\hypertarget{the-results}{%
\subsection{The results}\label{the-results}}

\href{./remez.html}{Remez Algorithm}

\hypertarget{some-conditions-for-the-analysis}{%
\subsection{Some conditions for the
analysis}\label{some-conditions-for-the-analysis}}

CG is doing the Lanczos algorithm in disguise. In particular,
normalizing the residuals from CG gives the vectors \(q_j\) produced by
the Lanczos algorithm, and combing the CG constants in the right way
gives the coefficients for the three term Lanczos recurrence.

The analysis by Greenbaum requires that the finite precision Conjugate
Gradient algorithm (viewed as the Lanczos algorithm) satisfy a few
properties. Namely,

\begin{itemize}
\tightlist
\item
  the three term Lanczos recurrence is well satisfied
\item
  the Lanczos vectors have norm close to one
\item
  successive Lanczos vectors are nearly orthogonal
\end{itemize}

As it turns out, nobody has actually ever proved that any of the
Conjugate Variant methods used in practice actually satisfy these
conditions (although some do numerically satisfy the conditions).

\hypertarget{communication-avoiding-conjugate-gradient-algorithms}{%
\section{Communication Avoiding Conjugate Gradient
Algorithms}\label{communication-avoiding-conjugate-gradient-algorithms}}

One of the main drawbacks to Conjugate gradient in a high performance
setting is \ldots{}..

\hypertarget{communication-bottlenecks-in-cg}{%
\subsection{Communication bottlenecks in
CG}\label{communication-bottlenecks-in-cg}}

Recall the standard Hestenes and Stifel CG implementation. In the below
description, every block of code after a ``\textbf{set}'' must wait for
the output from the previous block. Much of the algorithm is scalar and
vector updates which are relatively cheap (in terms of floating point
operations and communication). The most expensive computations each
iteration are the matrix vector product, and the two inner products.

\textbf{Algorithm.} (Hestenes and Stiefel Conjugate Gradient)
\begin{align*}
&\textbf{procedure}\text{ HSCG}( A,b,x_0 ) 
\\[-.4em]&~~~~\textbf{set } r_0 = b-Ax_0, \nu_0 = \langle r_0,r_0 \rangle, p_0 = r_0, s_0 = Ar_0, 
\\[-.4em]&~~~~\phantom{\textbf{set }}a_0 = \nu_0 / \langle p_0,s_0 \rangle
\\[-.4em]&~~~~\textbf{for } k=1,2,\ldots \textbf{:} 
\\[-.4em]&~~~~~~~~\textbf{set } x_k = x_{k-1} + a_{k-1} p_{k-1} 
\\[-.4em]&~~~~~~~~\phantom{\textbf{set }} r_k = r_{k-1} - a_{k-1} s_{k-1} 
\\[-.4em]&~~~~~~~~\textbf{set } \nu_{k} = \langle r_k,r_k \rangle, \textbf{ and } b_k = \nu_k / \nu_{k-1}
\\[-.4em]&~~~~~~~~\textbf{set }p_k = r_k + b_k p_{k-1}
\\[-.4em]&~~~~~~~~\textbf{set }s_k = A p_k
\\[-.4em]&~~~~~~~~\textbf{set }\mu_k = \langle p_k,s_k \rangle, \textbf{ and } a_k = \nu_k / \mu_k
\\[-.4em]&~~~~~\textbf{end for}
\\[-.4em]&\textbf{end procedure}
\end{align*}

A matrix vector product requires \(\mathcal{O}(\text{nnz})\) (number of
nonzero) floating point operations, while an inner product requires
\(\mathcal{O}(n)\) operations. For many applications of CG, the number
of nonzero entries is something like \(kn\), where \(k\) relatively
small. In these cases, the cost of floating point arithmetic for a
matrix vector product and an inner product is roughly the same. On the
other hand, the communication costs for the inner products are much
lower.

matvec is usually sparse

inner products require ``all reduce'' i.e.~collect information back from
all the different processors/nodes would like to be able to overlap
these as much as possible

\hypertarget{overlapping-inner-products}{%
\subsection{Overlapping inner
products}\label{overlapping-inner-products}}

We would like to be able to \emph{overlap} as many of the heavy
computations as possible. However, in the current form, we need to wait
for each of the previous computations before we are able to do a matrix
vector product or an inner product.

Using our recurrences we can write, \[
s_k = Ap_k = A(r_k + b_k p_{k-1}) 
= Ar_k + b_k s_{k-1}
\]

If we define the axillary vector \(w_k = Ar_k\), in exact arithmetic
using this formula for \(s_k\) will be equivalent to the original
formula for \(s_k\). However, we can now compute \(w_k\) as soon as we
have \(r_k\). Therefore, the computation of
\(\nu_k = \langle r_k,r_k \rangle\) can be overlapped with the
computation of \(w_k = Ar_k\).

These coefficient formulas seem to work better that CGCG..

\textbf{Algorithm.} (Chronopoulos and Gear Conjugate Gradient)
\begin{align*}
&\textbf{procedure}\text{ CGCG}( A,b,x_0 ) 
\\[-.4em]&~~~~\textbf{set } r_0 = b-Ax_0, \nu_0 = \langle r_0,r_0 \rangle, p_0 = r_0, s_0 = Ar_0, 
\\[-.4em]&~~~~\phantom{\textbf{set }}a_0 = \nu_0 / \langle p_0,s_0 \rangle
\\[-.4em]&~~~~\textbf{for } k=1,2,\ldots \textbf{:} 
\\[-.4em]&~~~~~~~~\textbf{set } x_k = x_{k-1} + a_{k-1} p_{k-1} 
\\[-.4em]&~~~~~~~~\phantom{\textbf{set }} r_k = r_{k-1} - a_{k-1} s_{k-1} 
\\[-.4em]&~~~~~~~~\textbf{set } w_k = Ar_k 
\\[-.4em]&~~~~~~~~\phantom{\textbf{set }} \nu_{k} = \langle r_k,r_k \rangle, \textbf{ and } b_k = \nu_k / \nu_{k-1}
\\[-.4em]&~~~~~~~~\textbf{set }\eta_k = \langle r_k, w_k \rangle, \textbf{ and } a_k = \nu_k / (\eta_k - (b_k/a_{k-1})\nu_k)
\\[-.4em]&~~~~~~~~\phantom{\textbf{set }} p_k = r_k + b_k p_{k-1}
\\[-.4em]&~~~~~~~~\phantom{\textbf{set }} s_k = w_k + b_k s_{k-1}
\\[-.4em]&~~~~~\textbf{end for}
\\[-.4em]&\textbf{end procedure}
\end{align*}

\textbf{Algorithm.} (Ghysels and Vanroose Conjugate Gradient)
\begin{align*}
&\textbf{procedure}\text{ CGCG}( A,b,x_0 ) 
\\[-.4em]&~~~~\textbf{set } r_0 = b-Ax_0, \nu_0 = \langle r_0,r_0 \rangle, p_0 = r_0, s_0 = Ar_0, 
\\[-.4em]&~~~~\phantom{\textbf{set }}w_0 = s_0, u_0 = Aw_0, a_0 = \nu_0 / \langle p_0,s_0 \rangle
\\[-.4em]&~~~~\textbf{for } k=1,2,\ldots \textbf{:} 
\\[-.4em]&~~~~~~~~\textbf{set } x_k = x_{k-1} + a_{k-1} p_{k-1} 
\\[-.4em]&~~~~~~~~\phantom{\textbf{set }} r_k = r_{k-1} - a_{k-1} s_{k-1} 
\\[-.4em]&~~~~~~~~\phantom{\textbf{set }} w_k = w_{k-1} - a_{k-1} u_{k-1}
\\[-.4em]&~~~~~~~~\textbf{set } \nu_k = \langle r_k,r_k\rangle, \textbf{ and } b_k = \nu_k/\nu_{k-1}
\\[-.4em]&~~~~~~~~\phantom{\textbf{set }} \eta_{k} = \langle r_k,w_k \rangle, \textbf{ and } a_k = \nu_k / (\eta_k - (b_k/a_{k-1})\nu_k)
\\[-.4em]&~~~~~~~~\phantom{\textbf{set }} t_k = Aw_k
\\[-.4em]&~~~~~~~~\textbf{set } p_k = r_k + b_k p_{k-1}
\\[-.4em]&~~~~~~~~\phantom{\textbf{set }} s_k = w_k + b_k s_{k-1}
\\[-.4em]&~~~~~~~~\phantom{\textbf{set }} u_k = b_k u_{k-1}
\\[-.4em]&~~~~~\textbf{end for}
\\[-.4em]&\textbf{end procedure}
\end{align*}

fda

\hypertarget{fdas}{%
\subsection{fdas}\label{fdas}}

\hypertarget{current-research-on-cg-and-related-krylov-subspace-methods}{%
\section{Current research on CG and related Krylov subspace
methods}\label{current-research-on-cg-and-related-krylov-subspace-methods}}

\hypertarget{avoiding-communication}{%
\subsection{Avoiding communication}\label{avoiding-communication}}

\begin{itemize}
\tightlist
\item
  CGCG, GVCG
\item
  \(s\)-step methods
\end{itemize}

\hypertarget{computing-matrix-functions}{%
\subsection{Computing matrix
functions}\label{computing-matrix-functions}}

\begin{itemize}
\tightlist
\item
  \(f(A)b\) for functions other that \(f(x) = x^{-1}\)
\end{itemize}
